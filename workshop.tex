%-----------------------------------------------------------------------------
%
%               Template for sigplanconf LaTeX Class
%
% Name:         sigplanconf-template.tex
%
% Purpose:      A template for sigplanconf.cls, which is a LaTeX 2e class
%               file for SIGPLAN conference proceedings.
%
% Guide:        Refer to "Author's Guide to the ACM SIGPLAN Class,"
%               sigplanconf-guide.pdf
%
% Author:       Paul C. Anagnostopoulos
%               Windfall Software
%               978 371-2316
%               paul@windfall.com
%
% Created:      15 February 2005
%
%-----------------------------------------------------------------------------


\documentclass[preprint]{sigplanconf}

% The following \documentclass options may be useful:

% preprint      Remove this option only once the paper is in final form.
% 10pt          To set in 10-point type instead of 9-point.
% 11pt          To set in 11-point type instead of 9-point.
% authoryear    To obtain author/year citation style instead of numeric.

\usepackage{amsmath,amssymb}
\usepackage[T1]{fontenc}
\usepackage{filecontents}
\usepackage{hyperref}

\begin{filecontents}{workshop.bib}
@Manual{Coq,
  title =        {The {Coq} Proof Assistant Reference Manual},
  author =       {{The Coq Development Team}},
  organization = {{LogiCal} Project},
  note =         {Version 8.0},
  year =         {2004},
  url =          "http://coq.inria.fr"
}

@InCollection{ProofGeneral,
  Title                    = {Proof General: A Generic Tool for Proof Development},
  Author                   = {Aspinall, David},
  Booktitle                = {Tools and Algorithms for the Construction and Analysis of Systems, {TACAS} 2000},
  Publisher                = {Springer Berlin Heidelberg},
  Year                     = {2000},
  Editor                   = {Graf, Susanne and Schwartzbach, Michael},
  Pages                    = {38--43},
  Series                   = {Lecture Notes in Computer Science},
  Volume                   = {1785},
  Doi                      = {10.1007/3-540-46419-0_3},
  ISBN                     = {978-3-540-67282-1},
  Language                 = {English},
  Url                      = {http://dx.doi.org/10.1007/3-540-46419-0_3}
}
\end{filecontents}

\begin{document}

\special{papersize=8.5in,11in}
\setlength{\pdfpageheight}{\paperheight}
\setlength{\pdfpagewidth}{\paperwidth}

\conferenceinfo{CoqPL 2016}{January 23, 2016, St. Petersburg, Florida, United States}
\copyrightyear{2016}
% \copyrightdata{978-1-nnnn-nnnn-n/yy/mm}
% \doi{nnnnnnn.nnnnnnn}

% Uncomment one of the following two, if you are not going for the 
% traditional copyright transfer agreement.

%\exclusivelicense                % ACM gets exclusive license to publish, 
                                  % you retain copyright

%\permissiontopublish             % ACM gets nonexclusive license to publish
                                  % (paid open-access papers, 
                                  % short abstracts)

\titlebanner{}        % These are ignored unless
\preprintfooter{TODO}   % 'preprint' option specified.

\title{Company-Coq: Taking Proof General one step closer to a real IDE}
\subtitle{A session on using Proof General and its new extension to write proofs more efficiently}

\authorinfo{Clément Pit-Claudel}
           {MIT CSAIL}
           {cpitcla@mit.edu}
\authorinfo{Pierre Courtieu}
           {CNAM, Lab. Cédric}
           {pierre.courtieu@cnam.fr}

\maketitle

\begin{abstract}
\texttt{Company-Coq} is a new Emacs package that extends \emph{Proof General} with a contextual auto-completion engine for Coq proofs and a many additional facilities to make writing proofs easier and more efficient. Beyond fuzzy auto-completion of tactics, options, module names, and local definitions, \texttt{company-coq} offers offline documentation and convenient snippets, plus display improvements and many other Coq-specific IDE features. The presentation will focus on a live demo of the system, with an emphasis on writing proofs in Emacs more efficiently.
\end{abstract}

\category{D.2.6}{Software Engineering}{Programming Environments---Integrated environments}

\terms Verification

\keywords IDE, documentation, proof engineering

\section*{Introduction}

Users of the Coq Proof Assistant \cite{Coq} are roughly divided between two interactive development environments\footnote{There also exists Coq interfaces for vim and Eclipse, though their use does not seem very widespread}: \emph{Proof General}, an extension of of Emacs written by David Aspinall et al. \cite{ProofGeneral}, and \emph{CoqIDE}, a Coq-specific development environment written from scratch by members of the Coq team and generally touted as more beginner-friendly (mostly due \emph{Proof General}'s dependence on Emacs). Both are powerful tools for writing proofs, and significantly improve the experience of proof authors when compared to Coq's simple read-eval-print loop. Yet these tools do not offer advanced features typically found in IDEs like Eclipse or Visual Studio, such as in-editor documentation or context-sensitive completion. In addition, when advanced features are in fact available (Proof General, for example, does supports snippets and improved display of mathematics), they tend to lack discoverability: users do not explore the menus and miss convenient features that would make them more efficient.

\section*{Overview of \texttt{company-coq}}

\texttt{Company-Coq} is a new Emacs package that attempts to fix some of these limitations: it extends \emph{Proof General} with many advanced IDE features (such as fuzzy completion, various Coq-specific snippets, and in-editor documentation for most of Coq's 2000-odd tactics, options, and errors), and it solves the discoverability issue by taking an ``all-on'' approach where the default distribution has most features automatically enabled. In addition, \texttt{company-coq} comes with a comprehensive tutorial that showcases most of its features. Some examples include:

\paragraph{Context-sensitive autocompletion with holes} Typing \texttt{applin} suggests \ldots

\paragraph{Lemma extraction}  \ldots

\paragraph{Point and click documentation} \ldots

\paragraph{Snippets} \ldots

Since it is not part of the core \emph{Proof General}, \texttt{company-coq} can be used as a convenient area to test new IDE features before they are merged into the main editor, and discuss new development directions: for example, although \emph{Proof General} does support enhanced displaying of mathematics (non-destructively displaying \texttt{fun (n m: nat) => forall p, p <> n -> p >= m} as \texttt{$\lambda$ (n m: $\mathbb N$) $\Rightarrow$ $\forall$ p, p $\neq$ n $\rightarrow$ p $\geq$ m}), few users seem to know about that feature. \texttt{Company-Coq}, on the other hand, enables a similar feature by default, and we have not heard of many users disabling or complaining about it.

\section*{Workshop description}

The workshop will consist in a quick run through \texttt{company-coq} features, and a tutorial on using these and other Proof General features to write proofs more efficiently. We will put emphasis on the lemma extraction feature in particular, showing how it can be used to structure proof developments more clearly. More generally, the workshop will also be a good occasion to showcase some features that \emph{Proof General} inherits from Emacs, and discuss how much of this work could be used to enhance other Coq interfaces. In particular, the development of \texttt{company-coq} has resulted in the manual annotation of the user manual to extract tactic templates and the associated documentation; the resulting list of tactics could be useful to other editors.

\bibliographystyle{abbrvnat}
\bibliography{workshop}

\end{document}
